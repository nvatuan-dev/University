\documentclass[a4paper]{report}

\usepackage{graphicx}
\usepackage[english]{babel}
\usepackage[utf8]{inputenc}
\usepackage[T1]{fontenc}
\usepackage{ragged2e}
\usepackage{hyphenat}
\usepackage{lmodern}
\usepackage{fancyhdr}
\usepackage[toc,page]{appendix}
\usepackage{tabularx}
\usepackage{float}

\begin{document}
    \centering
    \LARGE{\textsc{VIETNAM AVIATION ACADEMY}} \\
    \vspace{3mm}
    \normalsize{Department of Telecomunication - Electronics Engineering Technology} \\
    \vspace{3mm}
    \large{LOCATION IN HO CHI MINH CITY} \\
    \vspace{3mm}
    \includegraphics[scale=0.3]{download.jpg} \\
    \vspace{3mm}
    \normalsize{PROJECT REPORT:} \\
    \vspace{15mm}
    \huge{\textbf{"Circuit Remote Controlled Using Infarred Light"}} \\
    \vspace{20mm}
    \normalsize{Written by} \\
    \vspace{3mm}
    \large{\textit{Nguyen Van Anh Tuan}} \\ 
    \vspace{3mm}
    \textit{\large{Roll.No.1753020018}} \\
    \vspace{15mm}
    \textbf{\large{Under the guidance of}} \\
    \vspace{10mm}
    \centerline{\textbf{\large{Master Cao Xuan Kim Anh}}}

    % Lines down here to set footer and header. 
    \pagestyle{fancy}
    \fancyhf{}
    \rhead{Report Project 1}
    \lhead{CaptainJAV}
    \cfoot{\today}
    \renewcommand{\headrulewidth}{2pt}
    \renewcommand{\footrulewidth}{1pt}

    \newpage
    \centerline{\textbf{\huge{PREAMBLE}}} 
    \vspace{10mm}
    \begin{flushleft}
        In this day of advancement, we are indispensable remote control to control
        the devices we use every day as televisions, machines air conditioners, fan, etc.
        So how do remote controls work? Can control other objects in the distance? 
        Few know that the first remote control available during World War II. 
        Initially, people use RF technology (Radio Frequency) and then catch to start applying 
        IR (Infarred Remote) technology to the remote control. In today's life, we use 
        both types, however control remote use infarred in more often used. Let's see the principle operation 
        and construction of this remote control.
    \end{flushleft}
    \begin{flushright}
        \textbf{Auth. Nguyen Van Anh Tuan}
    \end{flushright}
    \thispagestyle{plain} % This command make this page have no header or footer.

    \newpage
    \tableofcontents

    \chapter{Introduction}
    \thispagestyle{fancy}
    \fancyhf{}
    \fancyhead[L]{CaptainJAV}
    \fancyhead[R]{Remote Controlled Using Infarred}
    \raggedright % Set all text to the left side.
    \rfoot{Page \thepage}
    \section{Preliminary introduce:} 
        With the current trend of modernization and industrialization, many modern technology 
        devices appear to help save time. We can mention as public technology of things 
        connected through the internet (Internet of Things) etc. But with expensive fees 
        are not suitable for the average consumer. From there, i founded simple solutions 
        with the same purpose and low cost.
        \linebreak
        \par In parallel, to supplement, to supplement the knowledge not studied 
        in school. From there, i selected "Remote Controlled Using Infarred" for the topic.
    \section{Objectives of the study:}
        To help reduce costs and supplement knowledge not researched at school.
    \section{Research Methods:}
        Find information on internet. \\
        Test on software. \\
        Construction circuit.


    \chapter{Find out theoretical related to the research}
    \thispagestyle{fancy}
    \fancyhf{}
    \fancyhead[L]{CaptainJAV}
    \fancyhead[R]{Remote Controlled Using Infarred}
    \rfoot{Page \thepage}
    \section{Application of remote controlled using infarred}
        Remote controlled now is using broadly, it use to controlled all wireless device. 
        Remotes and televisions are the best example for application of this recieve and transmitter 
        circuit. Or more application of this circuit. Beside that, we can see that remote controlled 
        can use with air conditioners, fans, or even use to turn on the lights in house...etc.
    \section{Define of Infarred (IR LED)}
        Infarred light (infarred ray) is the light we can't see it by our eyes, 
        they have wavelength from 700nm to 1mm. The infarred light have transmittion speed 
        is equal to lightspeed.
        \linebreak
        \par The infarred can transmit many signal channels. It is widely 
        applied in industry.
        \linebreak
        \par The amount of information that it can gain is 3 megabit/s. The 
        amount of information transmit with infarred light is many times larger 
        compared to the electromagnetic waves people still use.
        \linebreak
        \par Infarred rays are easily absorbed, poor penetration. In the word control 
        far by infarred, the beam emits a narrow, directed direction, so when recieve 
        must be in the right direction to use it.
        \linebreak
        \par Infarred wave have characteristics such as light (focusing through the lens, focal distance...). 
        Normal light and infarred light differ very clearly in light through the material.
        \linebreak
        \par Other than emitting invisible infarred light, IR Led look like a normal led and also
        works like a normal Led, it means it will consume 20mA and 3 Volts.
        \linebreak
        \par Besides that infarred is divided by wavelength into three main regions. However, follow the US classification is divided into 5 areas as follow: \\ 
        \vspace{3mm} 
        \begin{table}[ht]
            \centering
            \begin{tabular} { | p{1cm} | l | p{2cm} | p{2cm} | p{2cm} | p{3cm} | }
            \hline
            Name &Acronym &Wavelength &Frequency &Photon Energy &Featured \\ 
            \hline
            Near Infarred &NIR &750nm to 1.4$\mu$m &214-400THz &886-1653 meV &Determined by the absorption of water.Used in fiber optic telecomunications. \\ 
            \hline
            Short waves infarred &SWIR &1.4-3$\mu$m &100-214THz &413-886meV &Absorbed domestic increase significantly as of 1.45$\mu$m. Range 1.53-1.56$\mu$m is spectral region currently in use much in the far informed long road. \\ 
            \hline
            Medium waves infarred &MWIR &3-8$\mu$m &37-100THz &155-413meV &This band is called is thermal infarred, but it only detects slightly higher temperatures than body temperatures. \\ 
            \hline
            Long waves infarred &LWIR &8-15$\mu$m &20-37THz &83-155meV &This region is call "thermal infarred". \\ 
            \hline
            Far infarred &FIR &15-1000$\mu$m &0.3-20THz &1.2-83meV &See far infarred and far infarred laser. \\ 
            \hline
            \end{tabular}
            \caption{\label{tab:first}Classify Common of Infarred}
        \end{table}
    \section{Infarred receiver eye (TSOP-17xx)}
        It is an excellent line of infarred sensors for remote control applications. These infarred 
        sensors are designed to improve shielding electric interface. These devices are designed to 
        receive infarred rays from the infarred diode from a remote handset. 
        \linebreak
        \par TSOP 17xx is a part of the Photomodules family of infarred sensors modules miniature with PIN 
        photodiode and preamplification stage are placed in the shell epoxy. Its output is low and 
        gives +5V when off. Its output is demodulation to be able to decoded directly by the microprocessor. 
        Functions important modules include internal filter for PCM frequency capability. Compatible 
        with TTL and CMOS, low power consumption (5V and 5mA), immune to ambient light, anti-jamming, etc....
        \begin{table}[ht]
            \centering
            \begin{tabular} { | l | l | l | }
                \hline
                Number & Name & Description \\ \hline
                1 & Ground & Grounding \\ \hline
                2 & Vcc & Usually connect to +5V, maybe 6V \\ \hline
                3 & Signal & Output Signal \\
                \hline
            \end{tabular}
            \caption{\label{tab:second}Configuration of TSOP}
        \end{table}
        \end{document}