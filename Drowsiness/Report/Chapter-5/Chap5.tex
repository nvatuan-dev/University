\chapter{CONCLUSION AND TOPIC ORIENTATION}

\renewcommand{\headrulewidth}{0.5pt}
\renewcommand{\footrulewidth}{0.5pt}
\thispagestyle{plain}
\pagestyle{fancy}
\fancyhf{}
\fancyhead[L]{\textbf{CHAPTER 5}}
\fancyhead[R]{\textbf{DROWSINESS DETECTION AND ALERT SYSTEM IN THE CAR}}
\raggedright
\fancyfoot[L]{From: Nguyen Van Anh Tuan}
\fancyfoot[R]{Page \thepage}

\section{Conclusion}
    During the process of implementing this topic, I have built a system with practical applications for drivers 
    when participating in traffic, which is a sleep detection system, built on the Linux and Linux operating system. 
    Raspberry Pi 3B+ kit,with the support of OpenCV and Dlib libraries and programmed in Python programming language 
    with the task of issuing warnings when drivers doze off. \\ 
    \vspace{3mm}
    \textbf{Prodcut review:} \\ 
    To evaluate and comment objectively, I have tried to detect drowsiness in many different cases such as: front angle, 
    tilt angle, bright enough, low light, the results obtained are shown in the version below. 
    \begin{table}[ht]
        \centering
        \begin{tabular}{| l | l | l |}
            \hline
            \rowcolor{lightgray} Cases & Brightness enough & Lack of brightness \\ \hline
            Front corner & 91/100 images & 53/100 images \\ \hline
            Left tilt angle & 82/100 images & Almost undetectable \\ \hline
            Right tilt angle & 76/100 images & Almost undetectable \\ \hline
        \end{tabular}
        \caption{Results of eye state recognition in many cases}
    \end{table}
    Based on the above table, we can see that the identification results with relatively high accuracy in the case of 
    the frontal face in bright enough conditions, for the left and right tilt angles, the recognition process will be 
    more difficult. a bit, but the camera still catches the eye with high accuracy (in bright enough conditions) \\ 
    \vspace{3mm}
    On the contrary for the above three cases, the accuracy will gradually decrease in low light and almost undetectable 
    for the case of right tilt angle (low light conditions) and tilt angle greater than 40 degrees (both bright enough and low light).

\section{Limitations}
    \subsection{Advantages}
        \begin{itemize}
            \item Compact, convenient due to the small size of the Raspberry pi, it can be integrated directly on the car without 
            taking up too much space, can be integrated based on the existing security systems on the vehicle
            \item As a programmed application, it can save costs, the design and construction is also quite simple because the 
            hardware is simple, does not require complicated components
            \item Relatively high accuracy
            \item The program meets the actual requirements, the image processing time is fast, and the warning is issued in real time
        \end{itemize}
    \subsection{Disadvantages}
        \begin{itemize}
            \item Depends heavily on external conditions such as light, rotation angle,... 
            \item Face to face directly, if turning left or right with an angle greater than 40 degrees or the end of the head, 
            raising the head more than 30 degrees, it may not be detected
        \end{itemize}

\section{Development Orientation}
    \begin{itemize}
        \item Research using new, optimized and better recognition algorithms
        \item For the convenience of use, it is necessary to research and develop application software that can be controlled via smartphones
        \item Developing more diverse warning systems such as vibrating seats, or calling directly to the driver's phone, ... to wake up more effectively.
        \item When the driver shows signs of falling asleep, we can activate the car's self-driving mode and send a continuous warning to wake the driver.
    \end{itemize}